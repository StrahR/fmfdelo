\documentclass[mat1, tisk]{fmfdelo}
% \documentclass[fin1, tisk]{fmfdelo}
% Če pobrišete možnost tisk, bodo povezave obarvane,
% na začetku pa ne bo praznih strani po naslovu, …

%%%%%%%%%%%%%%%%%%%%%%%%%%%%%%%%%%%%%%%%%%%%%%%%%%%%%%%%%%%%%%%%%%%%%%%%%%%%%%%
% METAPODATKI
%%%%%%%%%%%%%%%%%%%%%%%%%%%%%%%%%%%%%%%%%%%%%%%%%%%%%%%%%%%%%%%%%%%%%%%%%%%%%%%

% - vaše ime
\avtor{Ime Priimek}

% - naslov dela v slovenščini
\naslov{Naslov dela}

% - naslov dela v angleščini
\title{Angleški prevod slovenskega naslova dela}

% - ime mentorja/mentorice s polnim nazivom:
%   - doc.~dr.~Ime Priimek
%   - izr.~prof.~dr.~Ime Priimek
%   - prof.~dr.~Ime Priimek
%   za druge variante uporabite ustrezne ukaze
\mentorica{izr.~prof.~dr.~Ime Priimek}
\somentor{doc.~dr.~Ime Priimek}

% - leto diplome
\letnica{2016} 

% - povzetek v slovenščini
%   V povzetku na kratko opišite vsebinske rezultate dela. Sem ne sodi razlaga
%   organizacije dela, torej v katerem razdelku je kaj, pač pa le opis vsebine.
\povzetek{V povzetku na kratko opišemo vsebinske rezultate dela. Sem ne sodi
razlaga organizacije dela -- v katerem poglavju/razdelku je kaj, pač pa le opis
vsebine.}

% - povzetek v angleščini
\abstract{Prevod slovenskega povzetka v angleščino.}

% - klasifikacijske oznake, ločene z vejicami
%   Oznake, ki opisujejo področje dela, so dostopne na strani https://www.ams.org/msc/
\klasifikacija{74B05, 65N99}

% - ključne besede, ki nastopajo v delu, ločene s \sep
\kljucnebesede{naravni logaritem\sep nenaravni algoritem}

% - angleški prevod ključnih besed
\keywords{natural logarithm\sep unnatural algorithm} % angleški prevod ključnih besed

% - angleško-slovenski slovar strokovnih izrazov
\slovar{
\geslo{continuous}{zvezen}
\geslo{uniformly continuous}{enakomerno zvezen}
\geslo{compact}{kompakten -- metrični prostor je kompakten, če ima v njem vsako zaporedje stekališče; podmnožica evklidskega prostora je kompaktna natanko tedaj, ko je omejena in zaprta  }
\geslo{glide reflection}{zrcalni zdrs ali zrcalni pomik -- tip ravninske evklidske izometrije, ki je kompozitum zrcaljenja in translacije vzdolž iste premice}  
\geslo{lattice}{mreža}  
\geslo{link}{splet}
\geslo{partition}{\textbf{$\sim$ of a set} razdelitev množice; \textbf{$\sim$ of a number} razčlenitev števila}
}

% - ime datoteke z viri (vključno s končnico .bib), če uporabljate BibTeX
\literatura{literatura.bib}

%%%%%%%%%%%%%%%%%%%%%%%%%%%%%%%%%%%%%%%%%%%%%%%%%%%%%%%%%%%%%%%%%%%%%%%%%%%%%%%
% DODATNE DEFINICIJE
%%%%%%%%%%%%%%%%%%%%%%%%%%%%%%%%%%%%%%%%%%%%%%%%%%%%%%%%%%%%%%%%%%%%%%%%%%%%%%%

% naložite dodatne pakete, ki jih potrebujete
\usepackage{algpseudocode}  % za psevdokodo
\usepackage{algorithm}      % za algoritme
\floatname{algorithm}{Algoritem}
\renewcommand{\listalgorithmname}{Kazalo algoritmov}

% deklarirajte vse matematične operatorje, da jih bo LaTeX pravilno stavil
% \DeclareMathOperator{\conv}{conv}
% na razpolago so naslednja matematična okolja, ki jih kličemo s parom
% \begin{imeokolja}[morebitni komentar v oklepaju] ... \end{imeokolja}
%
% definicija, opomba, primer, zgled, lema, trditev, izrek, posledica, dokaz

% za številske množice uporabite naslednje simbole
\newcommand{\R}{\mathbb R}
\newcommand{\N}{\mathbb N}
\newcommand{\Z}{\mathbb Z}
% Lahko se zgodi, da je ukaz \C definiral že paket hyperref,
% zato dobite napako: Command \C already defined.
% V tem primeru namesto ukaza \newcommand uporabite \renewcommand
\newcommand{\C}{\mathbb C}
\newcommand{\Q}{\mathbb Q}


%%%%%%%%%%%%%%%%%%%%%%%%%%%%%%%%%%%%%%%%%%%%%%%%%%%%%%%%%%%%%%%%%%%%%%%%%%%%%%%
% ZAČETEK VSEBINE
%%%%%%%%%%%%%%%%%%%%%%%%%%%%%%%%%%%%%%%%%%%%%%%%%%%%%%%%%%%%%%%%%%%%%%%%%%%%%%%

\begin{document}

\section{Uvod}

Na začetku prvega poglavja/razdelka (ali v samostojnem razdelku z naslovom
Uvod) napišite kratek zgodovinski in matematični uvod. Pojasnite motivacijo za
problem, kje nastopa, kje vse je bil obravnavan. Na koncu opišite tudi
organizacijo dela -- kaj je v kakšnem razdelku.

Če se uvod naravno nadaljuje v besedilo prvega poglavja, lahko nadaljujete z
besedilom v istem razdelku, sicer začnete novega. Na začetku vsakega
razdelka/podraz\-delka poveste, čemu se bomo posvetili v nadaljevanju. Pri
pisanju uporabljajte ukaze za matematična okolja, med formalnimi enotami
dodajte vezno razlagalno besedilo.

\section{Zveznost}

V tem poglavju bomo ponovili osnovne pojme, povezane z zveznostjo funkcij. Pri
tem bomo sledili~\cite{glob}.

Viri se v seznamu literature prikažejo le, če jih uporabimo v besedilu,
zato moramo citirati tudi~\cite{lang,zbornik,kalisnik,wiki,vec-avtorjev}.

\begin{definicija}
Funkcija $f\colon [a,b]\to\R$ je \emph{zvezna}, če \ldots
\end{definicija}

Podrobneje si poglejmo naslednji rezultat. Ker se bomo kasneje nanj sklicevali,
si pripravimo oznako z ukazom \verb|\label{oznaka}|. Seveda morajo biti oznake
različnih trditev različne. Enako označujemo tudi druga okolja oziroma enote
besedila.

\begin{izrek}\label{izr:enakomerno}
Zvezna funkcija na zaprtem intervalu je enakomerno zvezna.
\end{izrek}

\begin{dokaz}
Na začetku dokaza, če je to le mogoče in smiselno, razložite idejo dokaza.

Dokazovali bomo s protislovjem. Pomagali si bomo z definicijo zveznosti in s
kompaktnostjo intervala.  Izberimo $\varepsilon>0$. Če $f$ ni enakomerno
zvezna, potem za vsak $\delta>0$ obstajata $x, y$, ki zadoščata
\begin{equation}\label{eq:razlika}
  |x-y|<\delta\quad \text{in}\quad |f(x)-f(y)| \ge \varepsilon. \qedhere
\end{equation}
\end{dokaz}

V zgornjem primeru smo kvadratek za konec dokaza postavili v zadnjo vrstico
besedila, ki je vrstična formula, s pomočjo ukaza \verb|\qedhere|.  Ta ukaz
ustrezno deluje znotraj okolij \emph{equation, align*} in podobnih, ne pa
znotraj \verb|$$ ... $$|.

Oglejmo si še enkrat neenačbi~\eqref{eq:razlika}. Na formule se sklicujemo z
ukazom \verb|\eqref{oznaka}|, ki postavi zaporedno številko enačbe
v oklepaje, na trditve in ostale enote pa z ukazom \verb|\ref{oznaka}|. Črni
pravokotnik ob robu strani označuje predolgo vrstico, kjer \LaTeX ni uspel
pravilno postaviti besedila, zato mu morate pomagati, npr.\ tako, da stavek
nekoliko preoblikujete, sami razdelite nedeljivo enoto (npr. razdelite
matematično formulo na dva dela) ali pa ponudite možnosti za deljenje težavne
besede s pomočjo znakov\verb|\-|, ki jih postavite na mesto, kjer se besedo sme
deliti, npr.\  \verb|te\-žav\-nost\-ni\-ca|. Zgoraj bi tako lahko zapisali:

Oglejmo si še enkrat neenačbi~\eqref{eq:razlika}. Za sklicevanje na označene
enote besedila imamo na razpolago dva ukaza; na formule se sklicujemo z ukazom
\verb|\eqref{oznaka}|, \dots

V predzadnjem odstavku je v oklepaju za okrajšavo npr.\ nastal predolg
presledek sredi stavka, saj je \LaTeX zaradi pike sklepal, da je na tem mestu
stavka konec. Tak predolg presledek preprečimo tako, da za piko sredi stavka
postavimo poševnico in za njo presledek, torej \verb|\ |.\\

Če dokaz trditve ne sledi neposredno formulaciji trditve, moramo povedati, kaj
bomo dokazovali. To naredimo tako, da ob ukazu za okolje dokaz dodamo neobvezni
parameter,  v katerem napišemo besedilo, ki se bo izpisalo namesto besede
\emph{Dokaz}, npr.\ \verb|\begin{Dokaz}[Dokaz izreka \ref{izr:enakomerno}]|.

\begin{dokaz}[Dokaz izreka \ref{izr:enakomerno}]
  Dokazovanja te trditve se lahko lotimo tudi takole \ldots
\end{dokaz}

\subsection{Naslov morebitnega (pod)razdelka} Besedilo naj se nadaljuje v vrstici naslova, torej za ukazom \verb|\subsection{}| ne smete izpustiti prazne vrstice.

Podobno kot lahko spremenimo ime dokaza, lahko dodamo komentar v ime trditve,
torej s pomočjo neobveznega parametra; prvega od spodnjih izpisov dosežemo z
ukazom \verb|\begin{posledica}[izrek o vmesni vrednosti]|. Če želimo v
parametru navesti vir, pri katerem bomo navedli podatek o tem, kje v viru to
trditev najdemo, pa uporabimo ukaz
\verb|\begin{posledica}[\protect{\cite[izrek 3.14]{glob}}]|. Seveda lahko obe
možnosti kombiniramo.


\begin{posledica}[izrek o vmesni vrednosti]
  Naj bo $f$ zvezna in \ldots
\end{posledica}

Ali pa

\begin{posledica}[izrek o vmesni vrednosti \protect{\cite[izrek 3.14]{glob}}]
  Naj bo $f$ zvezna in \ldots
\end{posledica}

Podobno lahko napovemo tudi vsebino primera.

\begin{primer}[nezvezna funkcija nima nujno lastnosti povprečne vrednosti]
  Naj bo $f \colon \R \to \R$ dana s predpisom \dots
\end{primer}

\subsection{Pisanje algoritmov}
Za pisanje algoritmov sta na voljo okolji \texttt{algorithm} in
\texttt{algorithmic} iz paketov \texttt{algorithm} in \texttt{algorithmix}, ki
sodelujeta podobno kot \texttt{table} in \texttt{tabular}. Algoritmi plavajo
med tekstom, enako kot slike in tabele, nanje se lahko tudi sklicujemo, kot
prikazano v izvorni kodi in v algoritmu~\ref{alg:metoda}. Sklicujemo se lahko
tudi na pomembne vrstice, npr.\ na vrstico~\ref{alg:pomembna-vrstica}, ki
predstavlja glavni del algoritma. Za primer pisanja algoritma se posvetujte s
primerom v tem dokumentu, za bolj napredne primere uporabe, kot na primer
razbijanje algoritma na več kosov, pa z (precej razumljivo) uradno
dokumentacijo\footnote{\url{http://tug.ctan.org/macros/latex/contrib/algorithmicx/algorithmicx.pdf}}.
Če želite vključiti izvorno kodo nekega programa, priporočamo paket
\texttt{minted}\footnote{\url{https://github.com/gpoore/minted}}.

\algnewcommand\algorithmicto{\textbf{to}}
\algnewcommand\algorithmicin{\textbf{in}}
\algnewcommand\algorithmicforeach{\textbf{for each}}
\algrenewtext{For}[3]{\algorithmicfor\ #1 $\gets$ #2\ \algorithmicto\ #3\ \algorithmicdo}
\algdef{S}[FOR]{ForEach}[2]{\algorithmicforeach\ #1\ \algorithmicin\ #2\ \algorithmicdo}

\begin{algorithm}[ht]
  \caption{Opis, ki ima enako funkcionalnost kot opis pod sliko.}
  \label{alg:metoda}
  \raggedright
  \textbf{Vhod:} Števili $n, m \in \N, n > m$. \\
  \textbf{Izhod:} Decimalno število $x$, ki aproksimira rešitev enačbe $n x = m$.
  \begin{algorithmic}[1]
    \Function{reši}{$n$, $m$} \Comment{Vsi vhodni parametri morajo biti opisani.}
    \State $a \gets [\,]$ \Comment{Spremenljivka $a$ naj postane prazna kopica.}
    \For{$i$}{$1$}{$n$}
      \If{$i \operatorname{mod} 7 = 5$}
        \State \Call{heapop}{$a$}
      \ElsIf{$i < 5$}
      \State \Call{heappush}{$a, \frac{i+12}{7} + \pi$} \Comment{Lahko uporabljamo matematiko.}
      \Else
        \State \Call{heappush}{$a, i$}
      \EndIf
    \EndFor
    \Statex  \Comment{Prazna vrstica}
    \State $x \gets 0$  \Comment{To je primer komentarja.}
    \ForEach{e}{a}
      \State $x \gets 1 + \sqrt[e]{x}$
    \EndFor
    \While{$|x| > \varepsilon$}
      \State $x \gets x / 2$
    \EndWhile
    \State $x \gets m / n$ \label{alg:pomembna-vrstica}
    \State \Return $x$  \Comment{Vsi izhodni parametri morajo biti opisani nad algoritmom.}
    \EndFunction
  \end{algorithmic}
\end{algorithm}

\section{Konec dela}

Na konec dela sodita angleško-slovenski slovarček strokovnih izrazov in seznam
uporabljene literature, morebitne priloge (programska koda, daljša ponovitev
dela snovi, ki je bil obravnavan med študijem \dots) pa neposredno pred ti
enoti. Slovar naj vsebuje vse pojme, ki ste jih spoznali ob pripravi dela, pa
tudi že znane pojme, ki ste jih spoznali pri izbirnih predmetih. Najprej
navedite angleški pojem (ti naj bodo urejeni po abecedi) in potem ustrezni
slovenski prevod; zaželeno je, da temu sledi tudi opis pojma, lahko komentar
ali pojasnilo. Slovarska gesla navajajte z ukazom \verb|\geslo{}{}|, npr.\
\verb|\geslo{continuous}{zvezen}|.

Pri navajanju literature si pomagajte s spodnjimi primeri; najprej je opisano
pravilo za vsak tip vira, nato so podani primeri. Člen literature napišete z
ukazom \verb|\bibitem{oznaka} podatki o viru|, kjer mora \emph{ozmaka} enolično
določati vir.  Posebej opozarjam, da spletni viri uporabljajo paket url, ki je
vključen v~.cls datoteki. Polje ``ogled'' pri spletnih virih je obvezno; če je
kak podatek neznan, ustrezno ``polje'' seveda izpustimo. Literaturo je potrebno
urediti po abecednem vrstnem redu; najprej navedemo vse vire z znanimi avtorji
(tiskane in spletne) po abecednem redu avtorjev (po priimkih, nato imenih),
nato pa spletne vire brez avtorjev, urejene po naslovih strani. Če isti vir
navajamo v dveh oblikah, kot tiskani in spletni vir, najprej navedemo tiskani
vir, nato pa še podatek o tem, kje je dostopen v elektronski obliki.


\end{document}
